\documentclass[11pt,a4paper,sans]{moderncv}        % possible options include font size ('10pt', '11pt' and '12pt'), paper size ('a4paper', 'letterpaper', 'a5paper', 'legalpaper', 'executivepaper' and 'landscape') and font family ('sans' and 'roman')

% moderncv themes
\moderncvstyle{banking}                          % style options are 'casual' (default), 'classic', 'oldstyle' and 'banking'
\moderncvcolor{blue}                               % color options 'blue' (default), 'orange', 'green', 'red', 'purple', 'grey' and 'black'
%\renewcommand{\familydefault}{\sfdefault}         % to set the default font; use '\sfdefault' for the default sans serif font, '\rmdefault' for the default roman one, or any tex font name
%\nopagenumbers{}                                  % uncomment to suppress automatic page numbering for CVs longer than one page

% character encoding
%\usepackage[utf8]{inputenc}                       % if you are not using xelatex ou lualatex, replace by the encoding you are using
%\usepackage{CJKutf8}                              % if you need to use CJK to typeset your resume in Chinese, Japanese or Korean

% adjust the page margins
\usepackage[scale=0.75]{geometry}

% personal data
\name{Ekaterina}{Ovcharenko}
% \title{Developer}
\address{22, Zeleny pr., apt. 207}{Moscow}{Russia}
\phone[mobile]{+7915-257-6927}
\email{katerina.ovcharenko@gmail.com}


\newcommand{\kaspersky}{\href{kaspersky.com}{Kaspersky Lab (kaspersky.com)}}
% \newcommand{\responsibilities}{\emph{Key responsibilities}:}
% \newcommand{\achievements}{\hfill\emph{Achievements}:}

\newcommand{\responsibilities}[1]{ \emph{Responsibilities}:\begin{itemize} #1\end{itemize}}
\newcommand{\achievements}[1]{ \emph{Achievements}:\begin{itemize} #1\end{itemize}}


%----------------------------------------------------------------------------------
%            content
%----------------------------------------------------------------------------------


\begin{document}

\makecvtitle

% \section{Work Permit}
%     \hintstyle{Student Visa:} allowed to work for up to 20 hours each week\hfill 

\section{Skills}
    \cvitem{Computer languages}{Lua, Bash, Python, R, SQL, regular expressions, XML, HTML}
    \cvitem{Tools}{Flyspray, Chili Project, CVS, Git, SSH}
    \cvitem{Video editing}{Final Cut Pro 2, X}
    \cvitem{Systems}{Linux, Windows, MacOS}
    \cvitem{Soft skills}{managing group of 15 people, customer support, teaching students}
    


\section{Languages}
    \hintstyle{English:} Upper Intermediate\hfill 
    \hintstyle{Russian:} Native \hfill 


\section{Experience}

\cventry{May~2012 -- present}{Head of heuristic analysis group}{Kaspersky Lab}{Moscow}{} {
    \responsibilities {
        \item Leading a group of senior spam analysts.
        \item Providing antispam heuristics for Kaspersky line of products (Kaspersky Internet Security, Kaspersky Linux Mail Securilty, Kaspersky Security for Exchange, Kaspersky SDK).
        \item Guiding antispam engine developers. %Странно звучит. - Как переформулировать? Collaborating with ... for a улучшение продукта? Сюда доработка логики?
        \item Improving customer retention by adding a requested functionality and fine tuning the products.
        \item Supporting and developing a spam analysis toolkit (set of Perl, Lua and shell scripts).
        \item Blacklisting IPs by using user-based statistics (SQL). %Всё равно не понятно. - Что именно непонятно, какие слова надо пояснить?
        \item Training heuristics analysts
        \item Documentation % или updating instructions and manuals лучше?
        \item Supporting public tests %хуета какая-то. как это написать?
        \item The analysis of complex cases %непонятно как написать "с профилями" так, чтобы не написать "с профилями"
        % Ещё идеи ОБЯЗАННОСТЕЙ: ведение документации, обучение, прохождение публичных тестов, разбор сложных случаев и доработка логики продукта -- вспомни, что ещё ты делала вообще
        % 
    }
    \achievements {        % а скриптов каких-нибудь сюда не напихать?
        \item Increased the effective share of heuristics in the overall Spam detection statistics from 45\% to 70\% with:
        \begin{itemize}
            \item focusing and controlling efforts using a task management systems (FlySpray, Chilli Project);
            \item applying the latest approaches from IT conferences and academic papers; %Тебя сразу спросят, на каких конференциях была? - Это ты мне написал) Могу наврать про VB, например. Вообще я когда то имела в виду, что вся идея с Деденковскими луа-функциями про 16-ные загловки возникла, когда я прочитала про это статью в интернетиках.
            \item collaborating with a virus analysis group.
            \item %как можно сказать про совместную разработку всяких Довгополовских скриптов? всякие там сущности для бот-фермы, кластеризатор и вот это все. Может что-то про автоправила наврать? Хотя совсем врать плохо, придется подтверждать. Ничего кроме высосанных из пальца весов я для них не делала вроде бы.
        \end{itemize}
        \item trained 10+ analysts the fundamentals of Unix shell, ssh, CVS, GIT, regular expressions %спиздила из твоего резюме, не уверена, что надо, например
    }
}

\cventry{February~2010 -- April~2012}{Senior Spam Analyst}{Kaspersky Lab}{Moscow}{} {
    \responsibilities {
        \item Programming complex heuristics with heavy use of regular expressions.
        \item Monitoring spam traps data.
        \item Supporting a spam analysis toolkit (set of Perl, Lua and shell scripts).
        \item Supporting clients.
        \item Documentation
        \item Supporting public tests %хуета какая-то. как это написать?
    }
}

\cventry{January~2008 -- January~2010}{Spam Analyst}{Kaspersky Lab}{Moscow}{} {
    \responsibilities {
        \item Creating basic antispam heuristics.
    }
}

\cventry{June~2007 -- February~2008}{Support engineer}{Countrycom (telecommunication provider)}{Moscow}{} {
    \responsibilities {
        \item Resolving network connectivity issues including basic wiring tasks.
    }
}

\cventry{September~2001 -- April~2003}{Teacher of physics and mathematics}{Secondary school \#4}{Dolgoprudny}{} {
    \responsibilities {
        \item Running regular physics classes.
        \item Preparing students for tertiary entrance exams.
        \item Providing advanced lectures on physics and mathematics.
    }
}


\section{Education}

\cventry{2002 -- 2005}{Mathematics Teacher}{Moscow State Pedagogical University}{}{}{
    Notable attended courses:
    \begin{itemize}
        \item  Mathematical Analysis
        \item  Abstract Algebra
        \item  Analytic Geometry
        \item  Differential Geometry and Topology
        \item  Differential Calculus
        \item  Fundamentals of Probability Theory and Stochastic Processes
        \item  Mathematical Statistics
        \item  Discrete Mathematics
        \item  Mathematical Logic
        \item  Elementary Mathematics
    \end{itemize}
}  
\cventry{1997 -- 1999}{Applied Mathematics an Physics}{Moscow Institute of Physics and Technology}{}{}{
        Notable attended courses:
    \begin{itemize}
        \item Linear Algebra and Analytical Geometry
        \item Mathematical Analysis
        \item General Physics
        \item Theoretical Mechanics
        \item Complex Analysis
        \item General and Applied Economics
        \item Informatics and Application of Computers in Academic Researches
    \end{itemize}
}

\section{Interests}
online learning, video editing, kayaking, rafting, boardgames


% 
% \section{References}
% \begin{cvcolumns}
%   \cvcolumn{Category 1}{\begin{itemize}\item Person 1\item Person 2\item Person 3\end{itemize}}
%   \cvcolumn{Category 2}{Amongst others:\begin{itemize}\item Person 1, and\item Person 2\end{itemize}(more upon request)}
%   \cvcolumn[0.5]{All the rest \& some more}{\textit{That} person, and \textbf{those} also (all available upon request).}
% \end{cvcolumns}
% 


\end{document}

